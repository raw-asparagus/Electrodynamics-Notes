\documentclass[../main.tex]{subfiles}

\begin{document}

    \subsubsection{Vector Operations}
    Vectors - have direction and magnitude
    \begin{indented}
        i.e. displacment, velocity, acceleration, force \& momentum
    \end{indented}
    \begin{itemize}
        \renewcommand\labelitemi{--} 
        \item denoted as $\bm{A}$
        \item represented by arrows with length proportional to the magnitude of the vector and arrowhead indicating direction
        \begin{indented}
            e.g. Minus $\bm{A}$ ($-\bm{A}$) is a vector with the same magnitude as $\bm{A}$ but of opposite direction \texttt{(Figure 1.1-1)}
        \end{indented}
    \end{itemize}
    Scalars - have direction but no magnitude
    \begin{indented}
        e.g. mass, charge, density \& temperature
    \end{indented}
    \begin{itemize}
        \renewcommand\labelitemi{--} 
        \item denoted as $\abs{\bm{A}}$ or $A$
    \end{itemize}
    Vector operations:
    \begin{enumerate}
        \item Addition of two vectors\\
        Placing the tail of $\bm{B}$ at the head of $\bm{A}$ \texttt{(Figure 1.1-2)}, the resultant vector from the tail of $\bm{A}$ to the head of $\bm{B}$ is the sum - 
        \begin{eqnindent}
            \begin{flalign}
                \bm{A} + \bm{B} &&
            \end{flalign}
        \end{eqnindent}
        For the subtraction of two vectors, add its opposite \texttt{(Figure 1.1-3)}
        \begin{eqnindent}
            \begin{flalign}
                \bm{A} - \bm{B} = \bm{A} + \paren{-\bm{B}} &&
            \end{flalign}
        \end{eqnindent}
        \begin{itemize}
            \renewcommand\labelitemi{--}
            \item Commutative
            \begin{eqnindent}
                \begin{flalign}
                    \bm{A} + \bm{B} = \bm{B} + \bm{A} &&
                \end{flalign}
            \end{eqnindent}
            \item Associative
            \begin{eqnindent}
                \begin{flalign}
                    \paren{\bm{A} + \bm{B}} + \bm{C} = \bm{A} + \paren{\bm{B} + \bm{C}} &&
                \end{flalign}
            \end{eqnindent}
        \end{itemize}
        \item Multiplication by a scalar\\
        Multiplies the magnitude but leaves the direction unchanged \texttt{(Figure 1.1-4)}
        \begin{itemize}
            \renewcommand\labelitemi{--}
            \item if $a$ is -ve, the direction is reversed
            \item Distributive
            \begin{eqnindent}
                \begin{flalign}
                    a\paren{\bm{A} + \bm{B}} = a\bm{A} + a\bm{B} &&
                \end{flalign}
            \end{eqnindent}
        \end{itemize}
        \item Dot product of two vectors
        \begin{eqnindent}
            \begin{flalign}
                \bm{A} \cdot \bm{B} \equiv AB\cos\theta &&
            \end{flalign}
        \end{eqnindent}
        where $\theta$ is the angle $\bm{A}$ \& $\bm{B}$ form when placed tail-to-tail \texttt{(Figure 1.1-5)} and it results in a scalar. 
        \begin{itemize}
            \renewcommand\labelitemi{--}
            \item Commutative
            \begin{eqnindent}
                \begin{flalign}
                    \bm{A} \cdot \bm{B} = \bm{B} \cdot \bm{A} &&
                \end{flalign}
            \end{eqnindent}
            \item Distributive
            \begin{eqnindent}
                \begin{flalign}
                    \bm{A} \cdot \paren{\bm{B} + \bm{C}} = \bm{A} \cdot \bm{B} + \bm{A} \cdot \bm{C} &&
                \end{flalign}
            \end{eqnindent}
            \item Geometrically, $\bm{A} \cdot \bm{B}$ is the product of $A$ times the projection of $\bm{B}$ along $\bm{A}$
            \begin{eqnindent}
                \begin{flalign}
                    \begin{rcases}
                        \Rightarrow &\bm{A} \cdot \bm{B} = AB,\quad\text{if $\bm{A}$ \& $\bm{B}$ are parallel}\quad\\
                        &\bm{A} \cdot \bm{B} = 0,\quad\text{if $\bm{A}$ \& $\bm{B}$ are perpendicular}\quad\\
                    \end{rcases} &&
                \end{flalign}
            \end{eqnindent}
        \end{itemize}
        \item Cross product of two vectors
        \begin{eqnindent}
            \begin{flalign}
                \bm{A} \times \bm{B} \equiv AB\sin\theta~\hat{\bm{n}} &&
            \end{flalign}
        \end{eqnindent}
        where $\hat{\bm{n}}$ is a unit vector pointing perpendicular to the plane of $\bm{A}$ \& $\bm{B}$. 
        \begin{itemize}
            \renewcommand\labelitemi{--}
            \item with direction \texttt{(Figure 1.1-6)} resolved by the right-hand rule
            \begin{indented}
                i.e. the direction that of the thumb at when the fingers point towards the direction of the first vector while curling towards the second
            \end{indented}
            \item Geometrically, $\abs{\bm{A} \times \bm{B}}$ is the area of the parallelogram generated by $\bm{A}$ \& $\bm{B}$ \texttt{(Figure 1.1-6)}
            \item Distributive
            \begin{eqnindent}
                \begin{flalign}
                    \bm{A} \times \paren{\bm{B} + \bm{C}} = \paren{\bm{A} \times \bm{B}} + \paren{\bm{A} \times \bm{C}} &&
                \end{flalign}
            \end{eqnindent}
            \item Not Commutative
            \begin{eqnindent}
                \begin{flalign}
                    \paren{\bm{B} \times \bm{A}} = - \paren{\bm{A} \times \bm{B}} &&
                \end{flalign}
            \end{eqnindent}
        \end{itemize}
    \end{enumerate}

\end{document}
